%%%%%%%%%%%%%%%%%%%%%%%%%%%%%%%%%%%%%%%%%
% Medium Length Professional CV
% LaTeX Template
% Version 2.0 (8/5/13)
%
% This template has been downloaded from:
% http://www.LaTeXTemplates.com
%
% Original author:
% Trey Hunner (http://www.treyhunner.com/)
%
% Important note:
% This template requires the resume.cls file to be in the same directory as the
% .tex file. The resume.cls file provides the resume style used for structuring the
% document.
%
%%%%%%%%%%%%%%%%%%%%%%%%%%%%%%%%%%%%%%%%%

%----------------------------------------------------------------------------------------
%	PACKAGES AND OTHER DOCUMENT CONFIGURATIONS
%----------------------------------------------------------------------------------------

\documentclass{resume} % Use the custom resume.cls style
\usepackage{color,hyperref}
\usepackage[left=0.65in,top=0.5in,right=0.65in,bottom=0.575in]{geometry} % Document margins
\newcommand{\tab}[1]{\hspace{.2667\textwidth}\rlap{#1}}
\newcommand{\itab}[1]{\hspace{0em}\rlap{#1}}
\name{Abiy F. Melaku} % Your name

\usepackage{wasysym}
\usepackage[utf8]{inputenc}


\address{PhD Candidate, Department of Civil and Environmental Engineering, Western University}

\address{\href{mailto:amelaku@uwo.ca}{\textcolor {blue} {amelaku@uwo.ca}} \\ \href{tel:+12269776696} {\phone  \textcolor {blue} {+1 (226) 977-6696}}} % Your phone number and email

% \address{PhD Candidate, Department of Civil and Environmental Engineering, Western University} 



\usepackage[inline]{enumitem}
\usepackage[hang]{footmisc}
\setlength\footnotemargin{8pt}
\usepackage[usenames,dvipsnames]{color}
\renewcommand{\thefootnote}{\textcolor{red}{\arabic{footnote}}}
\usepackage{xurl}
\usepackage{footnotehyper}

\hypersetup{
    colorlinks=true,
    linkcolor=blue,
    filecolor=magenta,      
    urlcolor=blue,
    pdftitle={Sharelatex Example},
    bookmarks=true,
    }
    
    
%%%%%%%%%%%%%%%%%%%%%%%%%%%%%%%%%%%%%%%%%%%%
\begin{document}

%----------------------------------------------------------------------------------------
%	Research interest
%----------------------------------------------------------------------------------------

\begin{rSection}{Focus Areas}
\begin{itemize*}
  \item Numerical simulation of atmospheric boundary layer flows\\  
  \item Wind load evaluaton on buildings using CFD\\  
  \item Scientific computing \\  
  \item Fluid-structure interaction \\  

\end{itemize*}

\end{rSection}


\begin{rSection}{Education}
\begin{tabular}{r|p{14cm}}

\footnotesize{May. 2016 - Present} & \textcolor{black} {PhD Candidate in Civil Engineering and Scientific Computing, Western University (UWO), Canada} \\&\footnotesize{Dissertation topic}: \emph{A computational framework for unsteady aerodynamic and aeroelastic modeling of tall buildings under wind } \\&\footnotesize{Advisor}: Professor Girma Bitsuamlak (Western University)\multicolumn{2}{c}{}\\
 
\footnotesize{Feb. 2014 - Feb. 2016} & \textcolor{black} {MASc in Computational Structural Engineering, Chungbuk National University, South Korea} \\&\footnotesize{Thesis topic}: \emph{Fatigue assessment of intermittent fillet weld for vertical web stiffener of steel box girder bridge}\\&\footnotesize{Advisor}: Professor Jung Kyoung-Sub\multicolumn{2}{c}{} \\
 
 
\footnotesize{Sep. 2008 - Jun. 2013} & \textcolor{black} {BSc in Civil Engineering with Very Great Distinction, Adama Science and Technology University, Ethiopia} \\&\footnotesize{Final project topic}: \emph{Development of Structural Analysis and Design Software(ESADS) for Ethiopian Building Code of Standards(EBCS)}\\&\footnotesize{Advisors}: Dr. Beka Hailu and Eng. Ayele Zewdu\multicolumn{2}{c}{} \\

\end{tabular}
\end{rSection}

%%%%%%%%%%%%%%%%%%%%%%%%%%%%%%%%%%%%%%%%%%%%%%%%%%%%%%%%%

%%%%%%%%%%%%%%%%%%%%%%%%%%%%%%%%%%%%%%%%%%%%%%%%%%%%%%%

\begin{rSection}{Research Experience}
\begin{tabular}{r|p{14cm}}
\footnotesize{May. 2016 - Present} & \textcolor{black}{Graduate Research Assistant, Western University, Canada}\\& 
\footnotesize{Advisor}: \emp{Professor Girma Bitsuamlak}\\ & 
\footnotesize{Wind tunnel and CFD projects conducted\textcolor{red}{\footnotemark[1]}}\\ & 

\footnotesize  {Research project 1}: \emph{Development of a new inflow turbulence generation method for atmospheric boundary layer (ABL) flows and comparison with wind tunnel measurements} \\ & 
\footnotesize  {Research project 2\textcolor{red}{\footnotemark[2]}}: \emph{Preparation of aerodynamic database for the CAARC building conducting a wind tunnel test at the Boundary Layer Wind Tunnel Laboratory (BLWTL)} \\ & 
\footnotesize  {Research project 3}: \emph{Predicting the wind-induced response of the CAARC building using large-eddy simulation and dynamic analysis} \\ &  
\footnotesize  {Research project 4\textcolor{red}{\footnotemark[3]}}: \emph{Large-eddy simulation of wind loads on a tall building located in a city center} \\ &  
\footnotesize  {Research project 5}: \emph{Development of fluid-structure interaction framework for predicting the aeroelastic response of tall buildings under wind excitation} \\ &  

\end{tabular}
\end{rSection}

\footnotetext[1]{All of the listed research projects are part of my PhD study. In the research projects, I developed numerical modeles for wind load simulation and conducted wind tunnel tests for validating the models.}

\footnotetext[2]{Existing aerodynamic data sets for tall buildings are often not complete enough for validating CFD models. In this project, I conducted a wind tunnel test on the CAARC building and prepared detailed aerodynamic and wind field data for three exposure conditions. This database is finally used to validate the numerical models I developed.}

\footnotetext[3]{In collaboration with Thornton Tomasetti, this research project aimed to develope a numerical procedure to asses wind induced response of high-rise buildings with 500m radius proximity model. Experimental test results obtained from BLWTL were used to validate the numerical models.}


\pagebreak

\begin{rSection}{Scholarships and Awards}
\begin{tabular}{r|p{14cm}}
\footnotesize{May. 2016 - May. 2020} & \textcolor{black} {Ontario Trillium Scholarship\textcolor{red}{\footnotemark[4]}}\\& \footnotesize{Amount}: \emp{\$40,000/year} \\ \\

\footnotesize{Mar. 2012} & \textcolor{black} {ASTU Certificate of Academic Excellence}\\& \emp{Adama Science and Technology University, Adama, Ethiopia}\multicolumn{2}{c}{}\\

\footnotesize{Mar. 2010} & \textcolor{black} {AU Annual High Scoring Students’ Award,}\\& {Adama Science and Technology University, Adama, Ethiopia}\multicolumn{2}{c}{}

\end{tabular}
\end{rSection}


\begin{rSection}{Work Experience}
  \begin{tabular}{r|p{14cm}}
  
  \footnotesize{Sept. 2017 - Dec. 2017} & \textcolor{black} {Research Intern}\\& \footnotesize  {Company}: FM Global Research, Norwood, Massachusetts, USA\\& \footnotesize{Supervisor}: \emp{Dr. Lakshmana Doddipatla}\\ &  \footnotesize {Project title}\textcolor{red}{\footnotemark[5]}: \emph{Enhancing OpenFOAM’s wind engineering modeling capability}\multicolumn{2}{c}{}\\
  
  \footnotesize{Sept. 2019 - Dec. 2019} & \textcolor{black} {Research Intern}\\& \footnotesize  {Company}: FM Global Research, Norwood, Massachusetts, USA\\& \footnotesize{Supervisor}: \emp{Dr. Lakshmana Doddipatla}\\ &  \footnotesize {Project title}\textcolor{red}{\footnotemark[6]}: \emph{Large-eddy simulation of wind loads on roof-top equipment mounted on low-rise building}\multicolumn{2}{c}{}\\
  
  \footnotesize{July. 2014 - Feb. 2016} & \textcolor{black} {Construction Engineer (\emph{Part time})}\\ &  \footnotesize {Project site}: \emph{Namyangju-Byeolnae Road Construction Site(555-Gwangjeon-ri, Byeolnae-myeon, Namyangju-si, Gyeonggi-do, South Korea)} \\ &  \footnotesize  {Company}: Daelim Industrial Co. Ltd, Seoul, South Korea\multicolumn{2}{c}{}\\
  
  \end{tabular}
\end{rSection}

\footnotetext[4]{This project involves a C++ implementation of a ground surface and inflow boundary conditions in OpenFOAM that are particularly important for simulating wind loads on low-rise buildings using large-eddy simulation.}
\footnotetext[5]{I did a parametric study of the variation of wind loads on roof-top equipment depending on its elevation and location on the roof. The final results from the CFD simulations were properly validated against wind tunnel measurements and included in the Company's aerodynamic database.}

\footnotetext[6]{The Ontario Trillium Scholarships (OTS) program is an important initiative to attract top international students to Ontario,Canada for their PhD studies.}

% \pagebreak


% \pagebreak

\begin{rSection}{Refereed Journal Publications}

\begin{enumerate}

\item {\bf Melaku, A. F.} \& Bitsuamlak, G. T. (2021). A divergence-free inflow turbulence generator using spectral representation method for large-eddy simulation of ABL flows. \emph{Journal of Wind Engineering and Industrial Aerodynamics}.

\item {\bf Melaku, A. F.} \& Jung, K. S. (2017). Evaluation of welded joints of vertical stiffener to web under fatigue load by hotspot stress method. \emph{International Journal of Steel Structures}.

\item {\bf Melaku, A. F.}, Geleta, T. N. \& Jung, K. S. (2015). Application of object-oriented finite element method in structural mechanics. \emph{Journal of the Institute of Construction Technology}.

\end{enumerate}
\end{rSection}

\newpage


% \pagebreak

\begin{rSection}{Conference Publications}
\begin{enumerate}


\item {\bf Melaku, A. F.}, Doddipatla, L. S., Bitsuamlak \& G. T. (2021). Large-eddy wimulation of wind Loads on a roof-mounted cube: contribution to experimental database  In \emph{The 6th American Association for Wind Engineering Workshop}, Clemson University, Clemson, SC, USA.

\item Geleta, T. N., Elshaer, A., {\bf Melaku, A. F.} \& Bitsuamlak, G. T. (2018). Computational wind load evaluation of low-rise buildings with complex roofs using LES. In \emph{The 7th International Symposium on Computational Wind Engineering 2018.}, Seoul, Republic of Korea.  

\item {\bf Melaku, A. F.}, Bitsuamlak, G. T., Elshaer \& A., Aboshosha, H. (2017). Synthetic inflow turbulence generation methods for LES study of tall building aerodynamics. In \emph{The 13th Americas Conference on Wind Engineering (13ACWE)}, Gainesville, Florida, USA.  

\item  Adamek K., {\bf Melaku, A. F.}, Bitsuamlak, G. T. \& Sadeghpour, F. (2017). Wind safety assessment during high rise building construction. In \emph{The 13th Americas Conference on Wind Engineering (13ACWE)}, Gainesville, Florida, USA.  

\end{enumerate}
\end{rSection}

%%%%%%%%%%%%%%%%%%%%%%%%%%%%%%%%%%%%%%%%%%%%%%%%%%%%%%%%%%%%%%%%%

% \pagebreak

% \begin{rSection}{Book Chapters and Design Guides}
% \begin{enumerate}

% \item {\bf Bezabeh, M. A.}, Bitsuamlak, G. T. \& Tesfamariam, S., \& Popovski, M. (2020). Chapter 4.3.1: Analysis and design of tall timber buildings for wind loads. In the 2\textsuperscript{nd} edition of the \emph{Technical Guide for the Design and Construction of Tall Wood Buildings in Canada}\textcolor{red}{\footnotemark[14]}. Prepared for \emph{FPInnovations}, Vancouver, Canada.

% \end{enumerate}
% \end{rSection}


\begin{rSection}{Technical Reports}
\begin{enumerate}

\item{\bf Melaku, A. F.}, Geleta, T. N., Birhane, T. H.,  \& Bitsuamlak, G. T. (2021). Enabling OpenFOAM® for wind load evaluation. Prepared for \emph{FM Globa Research}, Norwood, Massachusetts, USA. 

\end{enumerate}
\end{rSection}


\begin{rSection}{Journal papers under preparation}
\begin{enumerate}

\item {\bf Melaku, A. F.}, Doddipatla, L. S. \& Bitsuamlak, G. T. Large-eddy simulation of wind loads on a rooftop equipment: assessment of the loading mechanism and the effect of equipment location. In preparation for \emph{Journal of Wind Engineering and Industrial Aerodynamics}.

\item {\bf Melaku, A. F.} \& Bitsuamlak, G. T. Computationally efficient simulation of multivariate stochastic processes using Nyström approximation. In preparation for \emph{Probabilistic Engineering Mechanics}.

\item {\bf Melaku, A. F.} \& Bitsuamlak, G. T. Predicting the wind induced responses of a high-rise building using LES and dynamic anaylysis: Validation with wind tunnel data. In preparation for \emph{Journal of Wind Engineering and Industrial Aerodynamics}.


\end{enumerate}
\end{rSection}


% \vspace{0.5cm}

\pagebreak

\begin{rSection}{Codes}
  \begin{tabular}{r|p{14cm}}
  
  \footnotesize{2018 - Present} & \textcolor{black} {Divergence-free Spectral Representation (DFSR)}\\& \footnotesize  {Github link}: \url{https://github.com/abiyfantaye/DFSR} \\& \footnotesize{Discription}: \emph{Computationally efficient inflow turbulence generation method developed for large-eddy simulation of the atmospheric boundary layer(ABL) flows. The method is developed targeting LES-based wind load evaluation application on structures.} \\ \\
  
  \footnotesize{2019 - Present} & \textcolor{black} {Pressure Integration Model(PIM)}\\& \footnotesize  {Github link}: \url{https://github.com/abiyfantaye/PIM} \\& \footnotesize{Discription}: \emph{A python script that analyses pressure data for buildings and report estimated wind loads and structural response. Compatable with wind tunnel and CFD data.} \\ \\

  \footnotesize{2020 - Present} & \textcolor{black} {Fluid-Structure Interaction(FSI) for tall building eroelastic analysis}\\& \footnotesize{Discription}: \emph{A C++ library to perform fluid-structure interaction using OpenFOAM's unsteady solver and in-house developed multi-degree of freedom(MDOF) structural solver. The code implements a direct load and displacement transfer mechanism for efficient computation.} \\ \\

  \footnotesize{2014 - 2015} & \textcolor{black} {Object-Oriented Finite Element Method(OOFEM)}\\& \footnotesize  {Github link}: \url{https://github.com/abiyfantaye/OOFEM} \\& \footnotesize{Discription}: \emph{An object-oriented finite element framework for 2D linearly elastic structural analysis which is capable of handling both static and transient analysis scenarios.} \\ \\
  
  \footnotesize{2012 - 2013} & \textcolor{black} {Ethiopian Structural Analysis and Design Software(ESADS)}\\& \footnotesize  {Github link}: \url{https://github.com/abiyfantaye/ESADS} \\& \footnotesize{Discription}: \emph{A structural analysis and design software developed based on the Ethiopian Building Code of Standards(EBCS-1995). ESADS can be used to analyze and design reinforced concrete structures such as beam, baxial column, slab and foundation footings.} \\ \\

  \footnotesize{2011 - 2012} & \textcolor{black} {Structural Analyis Program(STRAP)}\\& \footnotesize {Github link}: \url{https://github.com/abiyfantaye/STRAP} \\& \footnotesize{Discription}: \emph{A GUI bases structural analysis application developed mainly to analyze small-scale 2D structures like beam, truss and plane frames. This program was written during the 3rd-year of my undergraduate study.} \\ \\
  
  \end{tabular}
\end{rSection}


\begin{rSection}{HPC Training}
  \begin{tabular}{r|p{14cm}}
    \footnotesize{July 2016} & \textcolor{black}{Ontario High Performance Computing Summer School} \\& \footnotesize{University of Toronto, Ontario, Canada} \\& \\
    \footnotesize{July 2018} & \textcolor{black}{Compute Ontario Summer School on High Performance and Technical Computing } \\& \footnotesize{Western University, Ontario, Canada} \\& \\
  \end{tabular}
\end{rSection}


\begin{rSection}{Programing languages}
  \begin{tabular}{r|p{14cm}}
    \textcolor{black}{Languages} & \footnotesize{C, C++, C\#, Python, MATLAB} \\& \\
    \textcolor{black}{Parallel Programming} & \footnotesize{MPI, OpenMP, CUDA, OpenCL} \\& \\
    \textcolor{black}{Shell Programming} & \footnotesize{bash, tcsh} \\& \\
  \end{tabular}
\end{rSection}

\pagebreak
\begin{rSection}{Software Skills}
  \begin{tabular}{r|p{14cm}}
    \textcolor{black}{CFD} & \footnotesize{OpenFOAM, Star-CCM+} \\& \\
    \textcolor{black}{Grid generation} & \footnotesize{HEXPRESS, OpenFOAM snappyHexMesh} \\& \\
    \textcolor{black}{Structural analysis} & \footnotesize {SAP 2000, ETABS, Midas Civil, Abaqus FEA} \\& \\
    \textcolor{black}{CAD Modeling} & \footnotesize{AutoCAD, SALOME} \\& \\
    \textcolor{black}{Visualization} & \footnotesize{ParaView} \\& \\
  \end{tabular}
\end{rSection}


\begin{rSection}{Teaching Experience}
\begin{tabular}{r|p{14cm}}

\footnotesize{Jan. 2018 - Apr. 2018} & \textcolor{black} {Graduate Teaching Assistant, Western University}\\& \footnotesize{Course}: \emp{Computational Methods for Civil Engineering}\\&
\footnotesize{Course Mentor}: {Dr. Martha Dagnew}\multicolumn{2}{c}{}\\

\footnotesize{Sep. 2018 - Dec. 2018} & \textcolor{black} {Graduate Teaching Assistant, Western University}\\& \footnotesize{Course}: \emp{Computational Wind Engineering}\\&
\footnotesize{Course Mentor}: {Professor Girma Bitsuamlak}\multicolumn{2}{c}{}\\

\footnotesize{Sep. 2017 - Dec. 2017} & \textcolor{black} {Graduate Teaching Assistant, Western University}\\& \footnotesize{Course}: \emp{Engineering Statics}\\&
\footnotesize{Course Mentor}: {Dr. Ayman M. El Ansary}\multicolumn{2}{c}{}\\

\footnotesize{Sept. 2013 - Feb. 2014} & \textcolor{black} {Graduate Assistant Lecturer, Debre Berhan University, Ethiopia}\\& \footnotesize{Courses}: \emp{Engineering Mechanics I}\multicolumn{2}{c}{}

\end{tabular}
\end{rSection}

\begin{rSection}{Professional Membership}

\begin{rSubsection}{ }{}{}{}
\begin{enumerate}

\item Student member of Canadian Society for Civil Engineering (CSCE) Structures Division
\item Student member of American Society of Civil Engineers (ASCE) 
\item Member Graduate Engineering Society (GES) at Western University

\end{enumerate}
\end{rSubsection}
\end{rSection}

% \vspace{3cm}
\pagebreak

\begin{rSection}{REFERENCES}
\begin{rSubsection}{From Academia}{}{}{}
\begin{enumerate}


\item {Professor Girma Bitsuamlak, Ph.D., P.Eng., F CSCE \\
Canada Research Chair in Wind Engineering\\
Site-leader for Sharcnet - high performance computing center\\
Director (Research) Boundary Layer Wind Tunnel Laboratory\\
Director (Research) WindEEE Research Institute\\
Civil \& Environmental Engineering, Western University \\
ACEB Room# 4478 - London, ON, Canada \\
Email: \href{gbitsuam@uwo.ca}{\textcolor {blue} {gbitsuam@uwo.ca}}\\
Phone: \href{tel:+1 (519) 661-2111 x 88028}{\textcolor {blue} {+1 (519) 661-2111 x 88028}}\\ 
Website: \url{http://www.eng.uwo.ca/civil/faculty/bitsuamlak_g/index.html}}\\

\item {Professor Hassan Peerhossaini, PhD\\
Western Research Chair in Urban Resilience and Sustainability\\
Cross-appointed at Department of Mechanical \&  Materials Engineering, and \\
Civil \&  Environmental Engineering, Western University \\
ACEB Room# 4400C - London, ON, Canada \\
Email: \href{hpeerhos@uwo.ca}{\textcolor {blue} {hpeerhos@uwo.ca}}\\
Phone: \href{tel:+1 (519) 661-2111 x80319}{\textcolor {blue} {+1 (519) 661-2111 x80319}}\\ 
Website: \url{https://www.eng.uwo.ca/civil/faculty/peerhossaini_h/index.html}}\\
\end{enumerate}
\end{rSubsection}



\begin{rSubsection}{From Industry}{}{}{}
\begin{enumerate}
\item {Lakshmana Doddipatla, Ph.D.\\
Lead Research Scientist\\
FM Global Research\\
Email: \href{lakshmana.doddipatla@fmglobal.com}{\textcolor {blue} {lakshmana.doddipatla@fmglobal.com}}\\
Phone: \href{tel: +1 (781) 255-4988}{\textcolor {blue} {+1 (781) 255-4988}}}\\

\end{enumerate}
\end{rSubsection}
\end{rSection}
\end{document}
